\subsection{Machine Learning \--- Reformulation}
\newcommand{\N}{\textcolor{red}{TO\_REPLACE}}

\subsubsection{But}

\begin{frame}
\frametitle{But}
Les questions posées sont mises sont formes d'arbre. Mais qu'est-ce qui nous assure que cet arbre est le plus adapté ?
\pause
Le module formulation traduit cet arbre pour l'adapter au mieux au module wikidata.

Il change la forme et non le fond.
\end{frame}

\subsubsection{Fonctionnement}
\begin{frame}
\frametitle{Fonctionnement}
On projette notre arbre dans un $\mathbb{R}$-ev : l'espace des vecteurs.

\begin{defi}
Le dictionnaire lie un mot à un triplet de vecteurs.

On écrira $(m.s,m.p,m.o)$ le triplet du mot $m$.
\end{defi}

Si un mot m est à la place du sujet, prédicat ou objet on le confondra avec m.s, m.p ou m.o

\end{frame}

\begin{frame}
\frametitle{Projection}
\begin{tikzpicture}[x=2.5cm]
\node (s) at (1,0) {s};
\node (p) at (3,0) {p};
\node (r) at (2,1) {r};
\node (a) at (2,2) {a=contracter(s,r,p)};
\node (b) at (4,2) {b};
\node (l) at (3,3) {l};
\node (h) at (3,4) {h=contracter(a,l,b)};

\draw (s)--(a)--(p);
\draw (r)--(a);
\draw (a)--(h)--(b);
\draw (l)--(h);
\end{tikzpicture}
\end{frame}

\subsubsection{Retrouver une requête}

\begin{frame}
\frametitle{Du vecteur à l'arbre}
\begin{algorithm}[H]
\DontPrintSemicolon  % Some LaTeX compilers require you to use \dontprintsemicolon instead
\KwIn{$\delta>0$ et $a\in A$}
\KwOut{l'arbre de requête}
$(s,p,o) \gets \text{Decontracter}(a)$\;
Trouver m tel que $\N{m.s-s}<\delta$ est minimale \;
\lIf{m existe}{$s \gets m.s$}
\lElse{$s \gets C(\delta,s)$} 
Trouver m tel que  $\N{m.o-o}<\delta$ est minimale \;
\lIf{m exists}{$o \gets m.o$}
\lElse{$o \gets C(\delta,o)$} 
$p \gets \text{argmin}_m.r (\N{m.r-m.p})$\;
\Return{$(s,r,o)$}\;
\caption{ Du vecteur à l'arbre }
\end{algorithm}

\end{frame}

\subsubsection{Apprendre}

\begin{frame}
\frametitle{Entrainement}
Basé sur une distance sémantique, on calcule une distance entre le résultat réel et celui du module wikidata.

On utilise le graphe formé des relations ``instance de'' qui implique une norme facile a utiliser correspondant à la distance entre 2 nœuds.
\end{frame} 

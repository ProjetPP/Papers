\documentclass[a4paper,10pt]{article} % type, taille police

\usepackage[utf8]{inputenc} % encodage
\usepackage[T1]{fontenc} % encodage
\usepackage[french]{babel} % gestion du français
\usepackage{amssymb} % symboles mathématiques
\usepackage{textcomp} % flèche,  intervalle
\usepackage{stmaryrd} % intervalle entiers

\usepackage[left=3cm,right=3cm,top=3cm,bottom=3cm]{geometry} % marges
%\usepackage[hidelinks]{hyperref} % sommaire interactif dans un pdf
\usepackage[nottoc, notlof, notlot]{tocbibind} % affichage des références dans la table des matières (?)
\usepackage{float} % placement des figures
\usepackage[toc,page]{appendix} % ajout d'annexes
\usepackage{amsthm} % format des déf, prop...
\usepackage{amsmath} % matrices, ...
\usepackage{multirow} % fusionner cellules verticalement

%url cliquable et resizable
\usepackage[hyphens]{url}
\usepackage[hidelinks]{hyperref} 

%\usepackage[french,ruled,vlined,linesnumbered]{algorithm2e} % affichage d'algorithmes
\usepackage{tikz} % affichage de schémas
\usepackage{graphicx} % affichage d'images
\usepackage{url} % inclure des urls
\usepackage{bbold} % fonction caractéristique 1

%\Pbrovidecommand{\SetAlgoLined}{\SetLine} % paramètre pour algorithm2e
%\Pbrovidecommand{\DontPrintSemicolon}{\dontprintsemicolon}  % paramètre pour algorithm2e

\definecolor{bgreen}{rgb}{0.30,0.70,0}


%#############################################################################################################%
%#############################################################################################################%
%#############################################################################################################%

\title{Selected algorithms}
\author{ {\Large Yassine \textsc{Hamoudi}}}
\date{13 octobre 2014}

\begin{document}

\maketitle

%#############################################################################################################%
%#############################################################################################################%
%#############################################################################################################%

\section{Triple extraction from dependency graph}

The dependency graph is the best way to extract triples from questions using parsing tools. The parse structure tree is also often used in addition to the dependency graph.

~\\
The main tool about Stanford dependency tree is the \textit{Stanford typed dependencies manual}:
~\\
\begin{itemize}
	\item[] Marie-Catherine de Marneffe and Christopher D. Manning. \emph{Stanford typed dependencies manual}. 2013. \url{http://nlp.stanford.edu/software/dependencies_manual.pdf}
\end{itemize}


~\\
Some interesting topics:
\begin{itemize}
	\item \url{http://grokbase.com/t/opennlp/dev/139t2prbve/triplet-extraction-with-opennlp}
 	\item \url{http://stackoverflow.com/questions/9595983/tools-for-text-simplification-java}
	\item \sloppy \url{http://stackoverflow.com/questions/2712609/stanford-parser-traversing-the-typed-dependencies-graph?rq=1}
\end{itemize}

~\\
The objective is to understand enough the output of the Stanford parser (dependency tree) and then to implement an algorithm that extract triples from trees.

\section{Triple extraction from parse structure tree}

Using only the parse structure tree is far from being efficient, however some (very basics) algorithms exist:
~\\
\begin{itemize}
	\item[] Delia Rusu, Lorand Dali, Blaž Fortuna, Marko Grobelnik, Dunja Mladenić. \emph{Triplet extraction from sentences}. 2007. \url{http://ailab.ijs.si/delia_rusu/Papers/is_2007.pdf}
  \item[]
	\item[] Aaron Defazio. \emph{Natural Language Question Answering Over Triple Knowledge Bases}. 2009. \url{http://users.cecs.anu.edu.au/~adefazio/TripleQuestionAnswering-adefazio.pdf}
\end{itemize}

~\\
This approach can be implemented if the previous one (dependency tree) totally fail.

\section{From triples to database queries}

How mapping subject, predicate and objects to Wikidata entities. See part 2.2 of: 
~\\
\begin{itemize}
	\item[] Hakimov, Sherzod and Tunc, Hakan and Akimaliev, Marlen and Dogdu, Erdogan. \emph{Semantic Question Answering System over Linked Data Using Relational Patterns}. 2013. \url{http://edogdu.etu.edu.tr/wp-content/uploads/2005/01/2013-semantic-qa.pdf}
\end{itemize}


%#############################################################################################################%
%#############################################################################################################%
%#############################################################################################################%

\end{document}

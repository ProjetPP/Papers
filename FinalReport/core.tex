\section{Core}

As its name suggests, the core is the central point of the PPP. It is
connected to all other components — user interfaces and modules — through
the protocol defined above. It is developed in Python\footnote{\url{https://github.com/ProjetPP/PPP-Core/}}.

The core communicates with the user interfaces and the modules {\em via} HTTP:
each time the core receives a request from an interface, it forwards it
to modules using a configurable list of URL of other modules.

An example configuration is the one we use on the production server
(truncated):

\begin{verbatim}
{
    "debug": false,
    "modules": [
        {
            "name": "questionparsing_grammatical",
            "url": "python:ppp_questionparsing_grammatical.requesthandler:RequestHandler",
            "coefficient": 1,
            "filters": {
                "whitelist": ["sentence"]
            }
        },
        {
            "name": "cas",
            "url": "http://localhost:9003/cas/",
            "coefficient": 1,
            "filters": {
                "whitelist": ["sentence"]
            }
        },
        {
            "name": "spell_checker",
            "url": "python:ppp_spell_checker.requesthandler:RequestHandler",
            "coefficient": 1,
            "filters": {
                "whitelist": ["sentence"]
            }
        },
        {
            "name": "wikidata",
            "url": "http://wikidata.ppp.pony.ovh/",
            "coefficient": 1,
        },
        {
            "name": "datamodel_notation_parser",
            "url": "python:ppp_datamodel_notation_parser.requesthandler:RequestHandler",
            "coefficient": 1,
            "filters": {
                "whitelist": ["sentence"]
            }
        }
    ],
    "recursion": {
        "max_passes": 6
    }
}
\end{verbatim}

This example presents all the features of the core.

\subsection{Communicating with modules}

\begin{verbatim}
{
    "name": "cas",
    "url": "http://localhost:9003/cas/",
    "coefficient": 1,
    "filters": {
        "whitelist": ["sentence"]
    }
}
\end{verbatim}

This is the entry to connect the core to the CAS (Computer Algebra System)
module. We provide a nice name (to be
shown in logs), the URL to know how to reach it, a coefficient applied to
the measures, and filters (cf. chapter \ref{datamodel}).

\subsection{Recursion}

\begin{verbatim}
    "recursion": {
        "max_passes": 6
    }
\end{verbatim}

This is why the core is more than just a router. Not only does it forward
requests to modules, and forward their answers to the client, but it can
make several passes on a request.

Let's take an example: the question “What is the caiptal of India?”.
\footnote{\url{http://ppp.pony.ovh/?lang=en&q=What+is+the+caiptal+of+India\%3F}}
We need three passes to answer such a question.

\begin{enumerate}
    \item Spell-checking, to change it to “What is the capital of India?”
    \item Question parsing, to change it to a machine-understable query:
        “(India,capital,?)”
    \item Database request to Wikidata, to get the final answer:
        “New Delhi”
\end{enumerate}

After each pass, the router accumulates the different answers, until
there is no new answer. Then, it stops recursing and returns everything
to the client.
There is a fixed maximum number of passes, although it is not necessary
for now since modules we implemented cannot loop.

\subsection{Optimizations}

We implemented two optimizations in the core. Small experimentations showed that
it makes the core 25\% faster on queries with multiple passes.

The two of them are used in this example:

\begin{verbatim}
{
    "name": "questionparsing_grammatical",
    "url": "python:ppp_questionparsing_grammatical.requesthandler:RequestHandler",
    "coefficient": 1,
    "filters": {
        "whitelist": ["sentence"]
    }
}
\end{verbatim}

First, the filters. They can be
either a whitelist or a blacklist of root node types, that the core will
read to know if the module might do anything with such a tree.
Obviously, the question parsing modules cannot do anything with something
that is not a sentence, so it is useless to query it with something that
is not a sentence.

Secondly, the “python:” URL. The core can call directly Python modules
without using HTTP; and thus avoids the cost of serializing, using IPCs
(Inter-Process Communications), and deserializing (twice).
However, we need to run the module in the same process as the core, we therefore
only use it for light modules (questionparsing\_grammatical, spell-checker and OEIS).

\section{Libraries for modules}

We also provide a library in charge of handling and parsing HTTP
requests following the format defined in the data model. It was originally
used in the core, but we then splitted the code, as modules can reuse
it without a change.
This class is an abstraction over a Python HTTP library (python-requests),
allowing module developpers to focus on developping their actual code
instead of handling communication.

These has proven to be efficient, since the spell-checker and the OEIS modules
were created very quickly, using existing code.

It also exports its configuration library too, since modules share the
same way of handling configuration (a JSON file, whose path
is given {\em via} an environment variable).

For instance, here is the configuration file of the CAS module (Computer Algebra
System), used to allow system administrators to set memory and time limits
to computation:

\begin{verbatim}
{
    "max_heap": 10000000,
    "timeout": 10
}
\end{verbatim}

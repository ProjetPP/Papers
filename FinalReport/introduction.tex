The \emph{Projet Pensées Profondes} (PPP) provides a powerfull,
modular and open-source application to answer questions written in natural language.
We developed an eponymous set of tools that accomplish different tasks and fit
together thanks to a well defined protocol.

These various tasks include data querying (using the young and open
knowledge base Wikidata), question parsing (using machine learning and the
\CoreNLP software written by Stanford University),
requests routing, web user interface, feedback reporting, computer algebra, etc.

A deployment of these tools, \emph{Platypus}, is available online:

\begin{center}
    \url{http://ppp.askplatyp.us/}
\end{center}

The official website of the project is:

\begin{center}
    \url{http://projetpp.github.io/}
\end{center}

\medbreak

This report will firstly introduce the modular architecture of the
project and the datamodel used for communication between modules.

Then, all the modules actually developed in the PPP will be presented in details. Most of
them are part of the online deployment, \emph{Platypus}.

The user interface is composed of one web-page developed in HTML 5 with some
pieces of JavaScript and CSS \footnote{\url{https://github.com/ProjetPP/PPP-WebUI/}}.
We have taken care of having an
interface that fits nice on both huge screens of desktop computers
and small screens of phones using responsive design techniques.

\begin{figure}[!ht]
    \centering
    \includegraphics[width=1\textwidth]{WebUI.png}
    \caption{The user interface (Wikipedia header)}
\end{figure}

\begin{figure}[!ht]
    \centering
    \includegraphics[width=1\textwidth]{WebUI2.png}
    \caption{The user interface (OpenStreetMap location)}
\end{figure}

It is made of one huge text input with a button to submit
the query and another one to get a random question from a hand written list. The text area
allows to input question in natural language or using the data model notation like
\texttt{(Douglas Adam, birth date,?)} to find the
birth date of Douglas \textsc{Adam}. A parser written as a Python module
\footnote{\url{https://github.com/ProjetPP/PPP-datamodel-Python/blob/master/ppp\_datamodel/parsers.py}}
using
the PLY\footnote{\url{http://www.dabeaz.com/ply/}} library converts
the data model notation into the internal JSON representation.

When a query is submitted, the URL of the page is updated accordingly in order to allow the user to directly link to a given question. We use also an experimental web browser feature, Web Speech\footnote{\url{https://dvcs.w3.org/hg/speech-api/raw-file/tip/speechapi.html}} (fully implemented only in Chrome in December 2014), to allow the user to input the query using a speech input and, when this feature is used, the UI speaks a simplified version of the query results.

This interface relies on some famous libraries 
like jQuery\footnote{\url{http://jquery.com/}}, Bootstrap\footnote{\url{http://getbootstrap.com}}, MathJax\footnote{\url{http://www.mathjax.org/}} to display mathematics and Leaflet\footnote{\url{http://leafletjs.com/}} for maps.

\section{Logging}

All requests sent to the PPP are logged in order to be able to see what
the user are interested in and maybe use them as data set for the
question parsing modules that use Machine Learning. This logger
is powered by a backend that stores data
gathered via the user interface in a SQLite database.

The logger has a simple public API to retrieve last and top requests
in JSON (JavaScript Object Notation):
\url{http://logger.frontend.askplatyp.us/?order=last&limit=10} to get the
last 10 requests and \url{http://logger.frontend.askplatyp.us/?order=top&limit=5}
to get the top 5.

In order to use these data for machine learning we may also want, in the future,
to log user feedback in addition to the requests
themselves: after showing the user the way we interpreted their
question alongside the answer to their request, we would provide them a
way to give us feedback.
What we call feedback is actually a thumb up / thumb down pair of
buttons, and, if the latter is pressed, a way to correct the requests
parsing result so it can be fed to the Machine Learning algorithms.
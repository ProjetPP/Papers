The \emph{Projet Pensées Profondes} resulted in the creation of a query engine: \emph{Platypus}\footnote{\url{http://ppp.pony.ovh/}}.

This engine uses innovative techniques based on the analysis of the grammatical rules existing in natural languages. It enables it to answer correctly to non-trivial questions and to outperform the cutting-edge tools of the domain such as \emph{WolframAlpha} or \emph{Google Knowledge Graph} on some questions.

There still remains some improvements to add:
\begin{itemize}
    \item A better database. Contribute to the development of \emph{Wikidata} (e.g. to add the relevant measures). Add a module to retrieve information from an other database (e.g. \emph{Freebase}\footnote{\url{https://www.freebase.com/}}).
    \item A better question parsing. Improve the grammatical approach, by training the \emph{Stanford} library and refining our dependency analysis. Improve the machine learning approach, by increasing our dataset.
    \item Add new modules: cooking recipes, music knowledge, sport statistics, etc.
\end{itemize}

We produced a strong basis with a well structured architecture and a well defined data model. We are ready to welcome other developers in the \emph{Projet Pensées Profondes} to enhance \emph{Platypus}.

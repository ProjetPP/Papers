\section{Reformulation : a learning approach }

Efficiency of neural network nowadays can be very good. An approach based on network is also used here. But, the aim is not to learn answering question, there are two reasons for that: first the knowledge changes over time, e.g. the president name depends of the date, then something simple is wished, the learning process should also be continuous. In the second case, learning the answer implies to know proper Names which increase the size of the dictionary, that means the learning time. The module presented here try to reformulate a question. In fact it try to translate the question for the wikidata module.

\subsection{How it works}

There is a output of the grammatical module which builds a triple representation of the question. All work is done on this representation. The assumption ``all useful information is keep in this form''  is also necessary. The module has also a dictionary. There are four step to do. First a pretreatment replace proper names and numbers with tags, such that all word of the triple are in the dictionary, then the request is projected in the math space of request. Two last step reveres the two first, but the reverse projection should produce a better request, which mean the final request is adapted for the wikidata module.

\subsubsection{Mathematical spaces}

The mathematical space of request $\mathcal{R}$ is simply the subset of the $\mathbb{R}$ vector space of dimension 50, were all vector have an euclidean norm of 1.
A triple is formed of a subject request, a predicate and an object request, it is also natural to consider the triple over $\mathcal{R}$, and matching will alway follow the order subject, predicate, object.
Let us call vector the elements of $\mathcal{R}$.
The choice of a 50-dimension can be modified if necessary, but taking a higher dimension could slow the learning process, and with a lower space we could lost some expression power, which may lead to very poor results.

%To unify everything As we works with requests, we consider everything is a request, even a word.

%There are 2 generic spaces: the one of words which is a vector space of dimension 50 and the space of request which is the space of word triples. The first word of a triple represents the subject, the second represents the predicate and the last the object.
%To distinguish words which are vectors and words with letters, we will add the adjective English to the seconds.

\subsubsection{Dictionary}

The dictionary defines matching between English words and vectors triple, which is the base of the process. We use triple because a word could have different meaning depending on his nature, so predicate vector for a word is not the same as subject vector which is not the same as object vector.

\subsubsection{Pre- and post-treatment and treatment}

We evoke some reasons not to handle proper name directly (the dictionary size would increase and names could changes), there is another arguments: there is an arbitrary number of proper names, because you can invent some new names every time. That is why they are replace in a request by a tag NAMEi, $i\in{1,2,3}$. It allows us to have three different names in a question, and that is sufficient. The same happens for numbers. Tag UNKNOWN finally represents the ``holes'' in the request tree. At the end, after the treatment we replace tags with corresponding real names or numbers in the final tree.

Recognizing a number is not hard, in fact it is just checking it the characters sequence looks like numbers optionally followed by a dot and numbers. If the sequence is not a number or a ``hole'' and is not in the dictionary, the module treat it as a proper name.

\subsubsection{Project form a request to a vector}

The model allow an arbitrary request form, it mean we have a three with an unknown depth, and to keep the complete question information we should handle it so. But the request are triple, so it is easy to transform. First with the dictionary all word are replaced with vectors. Then we have to ascend the whole three to compute the root value.

Let define a matrix compact which take a triple of vector and merge them in one vector, here merging in not a intuitive operation, in fact we don't known how to merge this triple, that is why all coefficients of the matrix are unknown. To compute what we call a vector, output should be normalized.

Now by applying this operation bottom-up in the tree. Main idea is each node value represent the subtree request.


\subsubsection{Reconstruct a request from a vector}

This operation is quite symmetrical of the projection, with a matrix uncompact we can the same way obtain a vector triple from a vector, and recursively an tree appears. But the question is how to known if we should reapply the function uncompact or leave the vector as a leaf? First say in a triple predicate is never a tree, then object and subject will be let as leaf if a known word is near enough. Finally each node is replace with the nearest corresponding word of the dictionary, that mean for example for a vector in middle of a triple which is also a predicate take word with nearest predicates' vector.

Defining near enough is difficult. To avoid infinite loop we will take depth of nodes into account. We must take $\delta>0$ a precision factor and $g$ a growth factor. If $d$ is depth of node $n$, near enough means distance is bounded by $\delta*g^d$ with regard of euclidean norm.  

The algorithm is also :

\textsc{TreeReconstruction}
Entry $e$ is a vector

\begin{itemize}
\item (s,p,o)$\leftarrow$ uncompact(e)
\item Find word fp in the dictionary with nearest predicate to p
\item Find word fs in the dictionary with nearest subject to s
\item If distance fs to s is greater than $\delta$ fs $\leftarrow$ \textsc{TreeReconstruction}(s)
\item Find word fo in the dictionary with nearest subject to o
\item f distance fo to o is greater than $\delta$ fo $\leftarrow$ \textsc{TreeReconstruction}(o)
\item Return (fs,fp,fo)
\end{itemize}



\subsubsection{Remarks}

Matrix compact and uncompact are not bijective after restraining the request space to existent requests. Applying uncompact then compact should give the identity, with of course an error margin, but when compacting then uncompacting we only have to find an equivalent request, i.e. with the same meaning but with another formulation. If it were not the case it would produce exactly the same out as the input, that would be useless.

\section{Learning}

%\subsection{Advancement}

%We developed a Machine Learning module in C++\footnote{\url{https://github.com/ProjetPP/PPP-NLP-ML/}}.

%The implementation of the reformulation is written in C++. However it is not finished yet. The dictionary has been generated using the clex \footnote{\url{https://github.com/Attempto/Clex}}. The three functions are functional with a multithread approach to speed-up the computation time, also backpropagation is ready to be implemented.
%First approach of reformulation is implemented: that means taking a request tree in input, it returns another tree which should be equivalent if learning succeeded. However the learning process is not implemented yet.

%\subsection{Future work}

%Finding a way to learn everything with first or second approach is the most important, as the first answering module in functional, learning is possible. Then, learning and computation speed-up will be important, the search for nearest neighbor is long, maybe it is linear, but with near 100 000 words and high dimension it becomes consequent, use of heuristics could be a good idea, for example there exists distance sensitive hash. Kd-trees allow a search in log-time (with precomputation) ; but with dimension 50, the constant factor $2^{50}$ is too large. 


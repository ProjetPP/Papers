\documentclass[a4paper,10pt]{article}

\usepackage[utf8]{inputenc}
\usepackage[english]{babel}
\usepackage[T1]{fontenc}
\usepackage{palatino}

\usepackage[colorlinks=true]{hyperref}
\hypersetup{urlcolor=black,linkcolor=black}

\usepackage{enumerate}

\usepackage{fullpage}
\setlength{\parindent}{0pt}
\setlength{\parskip}{\medskipamount}

\usepackage{amsmath}
\usepackage{amssymb}
\usepackage{mathrsfs}
\usepackage{amsthm}
\usepackage{array}


\title{Project proposal}
\author{LIP - ENS Lyon}
\date{M1. 2014}

\begin{document}
\maketitle

\begin{tabular}{|ll|}
\hline
Title & Projet Pensées Profondes\\
Coordinator & Marc Chevalier\\
Members & Raphaël Charrondière, Marc Chevalier, Quentin Cormier, \\
        & Tom Cornebize, Yassine Hamoudi, Thomas Pellissier-Tanon\\
\hline
\end{tabular}

\section{Expected results}
\emph{What is the birth date of the first president of the United States?} This is
the typical question that we would like to answer automatically.

This requires three steps. 

Firstly, understanding the natural language. The input string given in standard 
English has to be transformed into a normal form. To do so, we will use natural 
language processing (NLP) libraries. We will also implement a machine learning (ML)
algorithm, in order to improve the answers of our software.

Then, the normal form is used to collect data, by querying some data base (e.g.
WikiData). Some operations may then be applied, like performing a sort.

Finally, the desired answer is displayed.


\section{Technological innovations}

\section{Concurrent services}

\section{Software}

\section{Target public}

\section{Schedule}

\section{Task partition}

\section{Organization}

\section{Budget}


\end{document}


Figure \ref{overview:gantt} presents the initial schedule of the project.

\begin{figure}[!ht]
\centering
\label{overview:gantt}
\caption{\textsc{Gantt} diagram of the project}
\begin{tikzpicture}[x=1cm, y=1cm, scale=0.8, every node/.style={scale=0.8}]
  \begin{ganttchart}%
  [today=10,today label=Midterm,
today label node/.append style=%
{anchor=north west},
today label font=\itshape\color{red},
today rule/.style=%
{draw=blue, ultra thick}, 
vgrid={*1{draw=none}, dotted},
inline
]{1}{24}
    \gantttitle{October}{9}\gantttitle{November}{9}\gantttitle{December}{6} \\
    \gantttitlelist{1,...,12}{2} \\
    \ganttbar[name = WP0]{Global organization}{1}{24} \\
    \ganttbar[name = WP1]{System administration}{1}{24} \\
    \ganttbar[name = WP2]{Soft. architecture}{1}{6} \\
    \ganttbar[name = WP3]{Communication}{1}{24} \\
    
    \ganttbar[name = WP5]{Router}{1}{6} \\
    \ganttlinkedbar[name = WP6]{Web UI}{7}{10}\\
    \ganttlinkedbar[name = WP7]{Web UI}{21}{24}\\
    \ganttbar[name = WP4]{Biblio.}{1}{3} \\
    \ganttbar[name = WP7]{Question Parsing: Grammatical Approach}{4}{24} \\
    \ganttbar[name = WP8]{Question parsing: Machine Learning}{4}{24} \\
    \ganttbar[name = WP9]{Wikidata module}{1}{24} \\
    \ganttbar[name = WP10]{Add-ons}{21}{24} \\
    \ganttlink[link mid=.4]{WP4}{WP7}
    \ganttlink[link mid=.2]{WP4}{WP8}

  \end{ganttchart}
\end{tikzpicture}
\end{figure}

Arriving to the midterm, we can see that only two work package are supposed to be finished. Indeed, the software architecture is well defined. On the other hand, the web user interface needs to have more features than we thought at the beginning.

A major reorganization is about machine learning. We split it into two different work package which implements two different machine learning algorithms.

In the other work package, the progression matches with what we expected.

Most of them produced, at least, a partially working code, allowing us to deploy the current state of the PPP on line (\url{http://ppp.pony.ovh}).

\chapter{Core}

\section{Communications}

As its name suggests, the core is the central point of the PPP. It is
connected to all other components — user interfaces and modules — through
the protocol defined above.

The core communicates with the user interfaces and the modules {\em via} HTTP:
each time the core receives a requests from an interface, it forwards it
to modules, using a configurable list of URL where to reach modules.

An example configuration is the one we use on the production server:

\begin{verbatim}
{
    "debug": false,
    "modules": [
        {
            "name": "nlp_classical",
            "url": "http://localhost:9000/nlp_classical/",
            "coefficient": 1
        },
        {
            "name": "flower",
            "url": "http://localhost:9000/flower/",
            "coefficient": 1
        },
        {
            "name": "wikidata",
            "url": "http://wikidata.ppp.pony.ovh/",
            "coefficient": 1
        }
    ]
}
\end{verbatim}

The current state is the the Core is successfully able with all modules
that have been written: the Wikidata module, an example Python module
(which answers the question  “Who are you?”), and the Natural Language
Processing module.

\section{Routing}

Besides from communicating with other pieces of the PPP, the Core will
also route requests in such a way the power of the modules can be
combined to give something greater.

For instance, when a user will input a question like “Who is the first
president of the United States?”, the Core will send the question to
all modules and get a parsed tree from the Natural Language Processing
module. Then, it will forward this answer to all other modules, which the
Wikidata module will be able to answer.

This part is not implemented, but will be next step in the implementation
of the Core.

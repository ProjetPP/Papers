\section{Communications}

As its name suggests, the core is the central point of the PPP. It is
connected to all other components — user interfaces and modules — through
the protocol defined above.

The core communicates with the user interfaces and the modules {\em via} HTTP:
each time the core receives a requests from an interface, it forwards it
to modules, using a configurable list of URL where to reach modules.

An example configuration is the one we use on the production server:

\begin{verbatim}
{
    "debug": false,
    "modules": [
        {
            "name": "nlp_classical",
            "url": "http://localhost:9000/nlp_classical/",
            "coefficient": 1
        },
        {
            "name": "flower",
            "url": "http://localhost:9000/flower/",
            "coefficient": 1
        },
        {
            "name": "wikidata",
            "url": "http://wikidata.ppp.pony.ovh/",
            "coefficient": 1
        }
    ]
}
\end{verbatim}

The current state is the Core is successfully able with all modules
that have been written: the Wikidata module, an example Python module
(which answers the question  “Who are you?”), and the Question parsing module.

\section{Libraries for modules}

The core also exports its class in charge of handling and parsing HTTP
requests following the format defined in the data model.
This class is an abstraction over a Python HTTP library (python-requests),
allowing module developpers to focus on developping their actual code
instead of handling communication.

These has proven to be efficient for connecting the Natural Language
Processing with the Core: we only had to copy-paste the demo code
(used for reading and printing in the console) into a function called
by this library, and it was working as is.


We are also planning on exporting the configuration
library too, since we notice modules are likely to share the
same way of handling configuration (a JSON file, whose path
is given {\em via} an environment variable.

\section{Routing}

Besides from communicating with other pieces of the PPP, the Core will
also route requests in such a way the power of the modules can be
combined to give something greater.

For instance, when a user will input a question like “Who is the first
president of the United States?”, the Core will send the question to
all modules and get a parsed tree from the Question parsing module. Then, it will forward this answer to all other modules, which the
Wikidata module will be able to answer.

This part is not implemented, but will be next step in the implementation
of the Core.

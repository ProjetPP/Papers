The project is about \textit{Question Answering}, a field of research included in \textit{Natural Language Processing} (NLP) theory. NLP mixes linguistic and computer science and it is made of all kinds of automated technics that have to deal with natural languages, such as French or English. For instance, automatic translation or text summarization are also parts of NLP.

In Natural Language Processing, sentences are often represented in a condensed and normalized form called \textit{triple representation}. It distinguishes three types of units: subject, predicate and object. These units are gathered into triples to catch the meaning of the sentence. For example, the phrase ``The turtle eats a salad.'' will be represented by the triple \hl{(the turtle,eats,the salad)}. Two triples can be associated to ``The president was born in 1950 and died in 2000.'': \hl{(the president, was born in, 1950)} and \hl{(the president,died in, 2000)}. This representation has been formalized into \textit{RDF} (Resource Description Framework) model. It consists of a general framework used for describing any Internet resource by sets of triples. Our first goal is to parse questions in order to get their triples representation.

Many algorithms have been developed since fifty years with the objective of understanding the syntax and the semantic of sentences. Two popular graph representations are widely used:
\begin{itemize}
 \item parse structure tree. It tries to split the sentence according to its grammatical structure.
 \item dependency tree. It reflects the grammatical relationships between words.
\end{itemize}
Existing libraries, such as NLTK\footnote{\url{http://www.nltk.org/}} or StanfordParser\footnote{\url{http://nlp.stanford.edu/software/lex-parser.shtml}} provide powerful tools to extract such representations.

We did not find a lot of article exposing procedures to get triples from the grammatical structure. For instance \cite{tripleparsetree} tries to perform this from the parse structure tree. However we believe the dependency tree could be more appropriate. We intend to develop a new algorithm using it.

We have also observed a growing use of machine learning technics in Natural Language Processing, and especially in Question Answering. Some very interesting results have been obtained with this popular field of computer science. We are trying to apply two existing machine learning algorithms to our subject.

Finally, some existing tools are very close to our goal\footnote{\url{http://quepy.machinalis.com/}}\textsuperscript{,}\footnote{\url{http://www.ifi.uzh.ch/ddis/research/talking.html}}. They allow us to have a clear idea of the state of the art, and what performances in question answering we can expect. Moreover,  some datasets of questions are available from two popular challenges in Question Answering (TREC and QUALD challenges). They will enable us to compare our performances to existing state of the art tools.

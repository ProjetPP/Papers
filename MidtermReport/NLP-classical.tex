\section{Grammatical approach}

\subsection{Stanford CoreNLP}

Our work relies on the \emph{Stanford CoreNLP} library
\footnote{\url{http://nlp.stanford.edu/software/corenlp.shtml}}, developped by
the Stanford Natural Language Processing group, which include members with 
linguistics or computer science backgrounds. 

Since this library is written in Java, we use a wrapper, written in Python
\footnote{\url{https://bitbucket.org/ProgVal/corenlp-python/overview}}.

These tools provide a server running in background, which takes a character string,
and returns a dependency tree, with some additional informations on certain words.
See figure \ref{tree_before} to have an overview of such a tree, for the sentence
\emph{What is the birth date of the first president of the United States of America?}.

The nodes of this tree are words, and the arcs are dependencies. For instance,
an arc $$\texttt{president}\xrightarrow{\texttt{det}}\texttt{the}$$ means that
\emph{the} is a determiner for \emph{president}. All possible dependencies are 
described in \cite{stanfordDep}. 

Some nodes of this tree are also endowed with tags. For instance, \emph{America}
has the tag \emph{location}.


\subsection{Preprocessing}


\subsection{Dependency analysis}

See figure \ref{tree_after}.


\subsection{Triple production}

See figure \ref{triple_conj}.

See figure \ref{triple_tree}.
